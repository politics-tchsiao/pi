\documentclass{article}
\usepackage[a4paper, left=2cm, right=2cm, top=2cm, bottom=2cm,textwidth=17cm,textheight=23cm]{geometry}
\usepackage{xeCJK}
\usepackage{setspace}
\usepackage{fontspec}
\usepackage{fancyhdr}
\usepackage{hyperref}
\usepackage{xcolor}
\setCJKmainfont{標楷體-繁}
\setmainfont{Times New Roman}
\pagestyle{fancy}
\fancyhf{}
\fancyfoot[C]{\textsf{\thepage}}
\renewcommand{\headrulewidth}{0pt}
\renewcommand{\footrulewidth}{0pt}
\title{\fontsize{16pt}{16pt}\selectfont Political Ideology}
\author{}
\date{}
\begin{document}
\setstretch{1.05}
\maketitle
\fontsize{14pt}{14pt}\selectfont
\begin{flushleft}
Instructor Information
\end{flushleft}
\begin{itemize}
\item Instructor: Dr. Tzu-Chi Hsiao. \\
\item Office: Room 506, Department of Political Science. \\
\item Office hours: 08:20-09:10, Every Wednesday. \\
\end{itemize}
Course Information \\
\begin{itemize}
\item Course: Political Ideology. \\
\item Credit: 2. \\
\item Hours: 14:20-16:10, Every Wednesday. \\
\item Prerequisite: Politics and Comparative Politics. \\
\item Course description \\
In this course, we will introduce subset of Political Science. Every conflicts or struggles are based on ideologies, you must learn the origin and solutions of classical ideologies. Last, each students need to find an interest ideology to present their ideas. \\ 
\item Course objectives \\
1. To understand the basic concepts of each ideologies. \\
2. To understand the basic concepts of each solutions. \\
3. To understand the basic concepts of each political systems. \\
\end{itemize}
\newpage
\begin{flushleft}
Grading \\
\end{flushleft}
\begin{itemize}
\item Midterm examination: 40\%. \\
\item Presentation: 60\%. \\
\end{itemize}
Textbook \\
\begin{itemize}
\item \textit{Political Ideology}, latest edition, written by Dr. Andew Heywood. \\
\end{itemize}
\end{document}
