\documentclass{article}
\usepackage[a4paper, left=2cm, right=2cm, top=2cm, bottom=2cm,textwidth=17cm,textheight=23cm]{geometry}
\usepackage{xeCJK}
\usepackage{setspace}
\usepackage{fontspec}
\usepackage{fancyhdr}
\usepackage{hyperref}
\usepackage{xcolor}
\setCJKmainfont{標楷體-繁}
\setmainfont{Times New Roman}
\pagestyle{fancy}
\fancyhf{}
\fancyfoot[C]{\textsf{\thepage}}
\renewcommand{\headrulewidth}{0pt}
\renewcommand{\footrulewidth}{0pt}
\title{Political Ideology}
\author{Tom T. Hsiao}
\date{}
\begin{document}
\setstretch{1.05}
\begin{center}
\fontsize{16pt}{16pt}\selectfont Political Ideology \\
\end{center}
\fontsize{14pt}{14pt}\selectfont
\begin{flushleft}
Instructor Information
\end{flushleft}
\begin{itemize}
\item Instructor: Tzu-Chi Hsiao. \\
\item Office: . \\
\item Office hours: 13:20-13:50, Every Mondays. \\
\end{itemize}
Course Information \\
\begin{itemize}
\item Course: Political Ideology. \\
\item Credit: 2. \\
\item Hours: 14:10-16:00, Every Mondays. \\
\item Prerequisite: Politics. \\
\item Course description \\
In this course, we will introduce subset of Political Science. Every conflicts or struggles are based on ideologies, you must learn the origin and solutions of classical ideologies. Last, each students need to find an interest ideology to present their ideas. \\ 
\item Course objectives \\
1. To understand the basic concepts of each ideologies. \\
2. To understand the basic concepts of each solutions. \\
3. To understand the basic concepts of each political systems. \\
\end{itemize}
\begin{flushleft}
Grading \\
\end{flushleft}
\begin{itemize}
\item Midterm examination: 40\%. \\
\item Presentation: 60\%. \\
\end{itemize}
Textbook \\
\begin{itemize}
\item Political Ideology, latest edition, written by Andew Heywood. \\
\end{itemize}
\end{document}
